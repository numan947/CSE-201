\documentclass{report}
\usepackage{hyperref,xcolor} %NEW PACKAGE
\hypersetup{
	colorlinks=true,
	linkcolor=blue,  	%SEE BEFORE
	urlcolor=purple
}

\newcommand{\xmpl}{\textbf{Example:}\\} %EXAMPLE EASE
\newcommand{\indnt}{\ \ \ \ } %MY INDENT 4 SPACES


\begin{document}

\title{The Elements of Style (Summary)}
\author{S.Mahmudul Hasan\\Roll: 1305043}
\date{\today}
\maketitle
\tableofcontents


\chapter{Introduction}
\raggedright  %NEW COMMAND
This is an offline given in our CSE 300 course titled `Technical Writing and Presentation'. We have been assigned the task of reading the book called The Elements of Style written by William Strunk JR. and E.B. White and write a summary of the whole book in about 1500 to 2500 words. The words with less than three characters as well as articles are not counted.\\
We have to use all the latex commands taught in the class at least once. We are instructed to write each of the rules in our own words and not to copy from the book. With each rule at least one example must be supplied and if there's any exception, we need to cover at least one exception with example. We have to come up with the examples on our own and we must not use the examples used in the book. We are instructed to use some introductory texts at the beginning of each of the chapters/sections.\\
The book which I am going to summarize i.e The Element of Style is a well known book which is used by a lot of professionals throughout the world. It is well reviewed by a lot of people in sites like \href{https://www.amazon.com/Elements-Style-William-Strunk-Jr/dp/1557427283}{Amazon}, \href{http://www.goodreads.com/book/show/33514.The_Elements_of_Style}{Goodreads}, \href{http://www.barnesandnoble.com/w/the-elements-of-style-william-strunk/1116794279}{Barns \& Noble} etc.
%NEW COMMAND
The book gives practical advice on improving writing skills. It gives emphasis on promoting a plain English Style. The book is regarded as a classic among the professionals.

\chapter{Elementary Rules of Usage}
\raggedright
LET HERE BE INTRODUCTIONS ar vallage na 

\section{Rule 1}
We need to \textbf{use 's to form possessive singular of nouns}. We should follow this rule without taking into account the final consonant.\\
\textbf{Example:}\\
Nafis's Samsung Galaxy S7\\
Numan's iPhone7\\
\subsection{Exception}
There're some exceptions to this rule. The exceptions are mostly related to the possessives of ancient proper names ending in \textit{-es} and \textit{-is}, the possessive \textit{Jesus'} and forms like \textit{for conscience' sake, for righteousness' sake}.\\
The possessives like \textit{hers, its, theirs, yours \emph{and} ours} are not written with apostrophe.
\section{Rule 2}
If there're three or more terms with single conjunction, we need to \textbf{use a comma after each term except the last one}. This type of comma is known as the ``serial'' comma.\\
\textbf{Example:}\\
Football, cricket, and volleyball\\
Anime, manga, or novel\\
The movie started, reached its peak, and left us overwhelmed.\\
\subsection{Exception}
This comma (serial comma) is not used in the names of companies or business firms.\\ \textbf{Example:}\\ 
Yukihira, Souma \& Nakiri

\section{Rule 3}
We should \textbf{enclose parenthetic expressions between commas}. It is applicable to dates too as they often contain parenthetic words or figures.\\
\textbf{Example:}\\
Going to walk in the morning, if you can, is good for your health.\\
March to December, 1971\\
\subsection{Exception}
There should be no comma separating a noun from a restrictive term of identification.\\
\textbf{Example:}\\
Jack the Ripper

\section{Rule 4}
We should use a \textbf{comma before a conjunction that introduces an independent clause}.\\ \textbf{Example:}\\
The golden age is gone now, and we have to wait a long time before getting another one.
\subsection{Exception}
Comma should be omitted if the relation between the two statements is close or immediate and are connected with \textit{and} but should be use if connected with \textit{but}.\\
\textbf{Example:}\\
I have done the compiler offline, but I still have homeworks to do.\\
He is healthy and has a lot of stamina.
\section{Rule 5}
We should not \textbf{join independent clauses with a comma}. If two or more clauses are complete without having any conjunction to join them, using semicolon is a good choice, using period is also good.\\
\textbf{Example:}\\
Oreki's plans are interesting; they're usually full of surprises.\\
\textit{or}\\
Oreki's plans are interesting. They're usually full of surprises.
\subsection{Exception}
A comma is used instead of a semicolon if the tone of the sentence is conversational.\\ \textbf{Example:}\\
He disagreed, I was ready for that.

\section{Rule 6}
We must not \textbf{break sentences in two}, i.e we should not use periods in place of commas.
\textbf{Example:}\\
While walking down the road. I saw a weak man. \textit{(wrong)}

While walking down the road, I saw a weak man. \textit{(correct)}

\section{Rule 7}
We should use \textbf{a colon} after an independent clause to introduce \textbf{ a list of particulars, an appositive, an amplification, or an illustrative quotation}. A colon is used to tell the reader that what follows is closely related to the preceding clause. It has more effect than the comma, less power to separate than the semicolon, and more formality than the dash. It is usually followed by an independent clause. The colon should not separate a verb from its complement or a preposition from its object.\\
We can join two independent clauses with a colon if the second one amplifies the first one.\\
We can use colon to introduce quotation supporting the topic of the sentence.
\textbf{Example:}\\
The recipe requires: a tomato, a full chicken, spice and mint.\\
They reached the hotel: it was still open late at night.\\
At the end of the day happiness lies in simple things like Steve Jobs said: ``Being the richest man in the cemetery doesn't matter to me. Going to bed at night saying we've done something wonderful, that's what matters to me.''

\section{Rule 8}
We should use \textbf{a dash to set off an abrupt break or interruption and to announce a long appositive or summary}. This is because a dash is stronger than a comma, less formal than a colon and more relaxed than parentheses.\\
\textbf{Example:}\\
It's okay---even though studying engineering is hard---it'll worth the pain someday.
\section{Rule 9}
There must be \textbf{equal} number of \textbf{subjects \emph{and} verbs}.\\ 
A plural verb should be used even in a relative clause following ``one of...''.\\
\xmpl
\indnt Robin, Tarek and their friends were playing cards.\\
A singular verb is used after \emph{each, either, everyone, everybody, neither, nobody, someone}.\\
\xmpl
\indnt Everybody lies the only variable is about what.\\
While using the word \emph{none}, singular verb is only used if the word means ``no one '' or ``not one'' and a plural verb is used when it suggests more than one thing or person.\\
\xmpl
\indnt none have the courage to do such a shameful deed.\\
\indnt None are allowed to go there at this late night.\\
A compound subject formed by two or more nouns joined by \emph{and} requires a plural verb.\\
\xmpl
\indnt The technician and the engineer were coming soon.\\
A singular subject doesn't become plural if other nouns are connected with it using \emph{with, as, as well as, in addition to, except, together with \emph{and} no less than.}\\
\xmpl
\indnt He as well as his brother was a gentleman.\\
\subsection{Exception}
Some nouns may appear plural but are usually used as singular and given a singular verb. This case is usually occur with idioms.\\
\textbf{Example:}\\
\indnt Powerful economics is needed to make a powerful country.

\section{Rule 10}
We must use \textbf{the proper case of pronoun}. The pronoun \textit{who} and the personal pronouns change with the change of subject or object.\\
\xmpl
\indnt Who goes there?\\
\indnt He who is enlightened is sure to succeed.\\
A pronoun in a comparison is nominative if it is the subject of a stated or understood verb. We should avoid ``understood'' verbs by supplying them.
\xmpl
\indnt He plays better than I. \indnt \indnt He plays better than I do.

\section{Rule 11}\label{rl11}
When using a participial phrase at the beginning of a sentence, it must refer to the grammatical subject. Violating this rule may often make sentences ludicrous.\\
\xmpl
\indnt Approaching slowly to the crowded place, I found that an exhibition is going on.

\end{document}