\documentclass[12pt]{report}
\usepackage{hyperref,xcolor,color,titlesec} %NEW PACKAGE
\hypersetup{
	colorlinks=true,
	linkcolor=blue,  	%SEE BEFORE
	urlcolor=brown
}

\titleformat{\chapter}{\normalfont\huge}{\thechapter.}{20pt}{\huge\it} %FORMAT THE CHAPTER HEADER STYLE

\newcommand{\xmpl}{\textbf{Example:}\\} %EXAMPLE EASE
\newcommand{\indnt}{\ \ \ \ } %MY INDENT 4 SPACES


\begin{document}
\title{\textbf{The Elements of Style (Summary)}}
\author{\textcolor{blue}{S.Mahmudul Hasan}\\ \textcolor{blue}{Roll: 1305043}}
\date{\textcolor{blue}{\today}}% 30th September,2016
\maketitle
\tableofcontents


\chapter{Introduction}\label{INTRO}

\raggedright  %NEW COMMAND
This is an \textit{offline} given in our \textbf{CSE 300} course titled \textit{\textbf{`Technical Writing and Presentation'}}. We have been assigned the task of reading the book called \emph{The Elements of Style} written by \emph{William Strunk JR. \emph{and} E.B. White} and write a summary of the whole book in about \textbf{1500 to 2500 words}. The words with less than \textbf{three} characters as well as articles are not counted.\\
We have to use all the \LaTeX commands taught in the class at least once. We are instructed to write each of the rules in our own words and not to copy from the book. With each rule at least one example must be supplied and if there's any exception, we need to cover at least one exception with example. We have to come up with the examples on our own and we must not use the examples used in the book. We are instructed to use some introductory texts at the beginning of each of the chapters/sections.\\
The book which I am going to summarize \textit{i.e The Element of Style} is a well known book which is used by a lot of professionals throughout the world. It is well reviewed by a lot of people in sites like \textbf{\href{https://www.amazon.com/Elements-Style-William-Strunk-Jr/dp/1557427283}{Amazon}},\textbf{\href{http://www.goodreads.com/book/show/33514.The_Elements_of_Style}{Goodreads}},\textbf{\href{http://www.barnesandnoble.com/w/the-elements-of-style-william-strunk/1116794279}{Barns \& Noble}} etc.
%NEW COMMAND
The book gives practical advice on improving writing skills. It gives emphasis on promoting a \textbf{plain \textit{English Style}}. The book is regarded as a \emph{classic} among the professionals.




\chapter{Elementary Rules of Usage}\label{ERU}



\raggedright
In this chapter some elementary rules of using English are taught and some common mistakes are pointed out. This is done by using examples and describing them.

\section{Rule 1}
We need to \textbf{use 's to form possessive singular of nouns}. We should follow this rule without taking into account the final consonant.\\
\textbf{Example:}\\
Nafis's Samsung Galaxy S7\\
Numan's iPhone7\\

\subsection{Exception}
There're some exceptions to this rule. The exceptions are mostly related to the possessives of ancient proper names ending in \textit{-es} and \textit{-is}, the possessive \textit{Jesus'} and forms like \textit{for conscience' sake, for righteousness' sake}.\\
The possessives like \textit{hers, its, theirs, yours \emph{and} ours} are not written with apostrophe.


\section{Rule 2}
If there're three or more terms with single conjunction, we need to \textbf{use a comma after each term except the last one}. This type of comma is known as the ``serial'' comma.\\
\textbf{Example:}\\
Football, cricket, and volleyball\\
Anime, manga, or novel\\
The movie started, reached its peak, and left us overwhelmed.\\

\subsection{Exception}
This comma (serial comma) is not used in the names of companies or business firms.\\ \textbf{Example:}\\ 
Yukihira, Souma \& Nakiri


\section{Rule 3}
We should \textbf{enclose parenthetic expressions between commas}. It is applicable to dates too as they often contain parenthetic words or figures.\\
\textbf{Example:}\\
Going to walk in the morning, if you can, is good for your health.\\
March to December, 1971\\

\subsection{Exception}
There should be no comma separating a noun from a restrictive term of identification.\\
\textbf{Example:}\\
Jack the Ripper


\section{Rule 4}
We should use a \textbf{comma before a conjunction that introduces an independent clause}.\\ \textbf{Example:}\\
The golden age is gone now, and we have to wait a long time before getting another one.

\subsection{Exception}
Comma should be omitted if the relation between the two statements is close or immediate and are connected with \textit{and} but should be use if connected with \textit{but}.\\
\textbf{Example:}\\
I have done the compiler offline, but I still have homeworks to do.\\
He is healthy and has a lot of stamina.


\section{Rule 5}
We should not \textbf{join independent clauses with a comma}. If two or more clauses are complete without having any conjunction to join them, using semicolon is a good choice, using period is also good.\\
\textbf{Example:}\\
Oreki's plans are interesting; they're usually full of surprises.\\
\textit{or}\\
Oreki's plans are interesting. They're usually full of surprises.

\subsection{Exception}
A comma is used instead of a semicolon if the tone of the sentence is conversational.\\ \textbf{Example:}\\
He disagreed, I was ready for that.


\section{Rule 6}
We must not \textbf{break sentences in two}, i.e we should not use periods in place of commas.
\textbf{Example:}\\
While walking down the road. I saw a weak man. \textit{(wrong)}

While walking down the road, I saw a weak man. \textit{(correct)}


\section{Rule 7}
We should use \textbf{a colon} after an independent clause to introduce \textbf{ a list of particulars, an appositive, an amplification, or an illustrative quotation}. A colon is used to tell the reader that what follows is closely related to the preceding clause. It has more effect than the comma, less power to separate than the semicolon, and more formality than the dash. It is usually followed by an independent clause. The colon should not separate a verb from its complement or a preposition from its object.\\
We can join two independent clauses with a colon if the second one amplifies the first one.\\
We can use colon to introduce quotation supporting the topic of the sentence.
\textbf{Example:}\\
The recipe requires: a tomato, a full chicken, spice and mint.\\
They reached the hotel: it was still open late at night.\\
At the end of the day happiness lies in simple things like Steve Jobs said: ``Being the richest man in the cemetery doesn't matter to me. Going to bed at night saying we've done something wonderful, that's what matters to me.''


\section{Rule 8}
We should use \textbf{a dash to set off an abrupt break or interruption and to announce a long appositive or summary}. This is because a dash is stronger than a comma, less formal than a colon and more relaxed than parentheses.\\
\textbf{Example:}\\
It's okay---even though studying engineering is hard---it'll worth the pain someday.


\section{Rule 9}
There must be \textbf{equal} number of \textbf{subjects \emph{and} verbs}.\\ 
A plural verb should be used even in a relative clause following ``one of...''.\\
\xmpl
\indnt Robin, Tarek and their friends were playing cards.\\
A singular verb is used after \emph{each, either, everyone, everybody, neither, nobody, someone}.\\
\xmpl
\indnt Everybody lies the only variable is about what.\\
While using the word \emph{none}, singular verb is only used if the word means ``no one '' or ``not one'' and a plural verb is used when it suggests more than one thing or person.\\
\xmpl
\indnt none have the courage to do such a shameful deed.\\
\indnt None are allowed to go there at this late night.\\
A compound subject formed by two or more nouns joined by \emph{and} requires a plural verb.\\
\xmpl
\indnt The technician and the engineer were coming soon.\\
A singular subject doesn't become plural if other nouns are connected with it using \emph{with, as, as well as, in addition to, except, together with \emph{and} no less than.}\\
\xmpl
\indnt He as well as his brother was a gentleman.\\
\subsection{Exception}
Some nouns may appear plural but are usually used as singular and given a singular verb. This case is usually occur with idioms.\\
\textbf{Example:}\\
\indnt Powerful economics is needed to make a powerful country.


\section{Rule 10}
We must use \textbf{the proper case of pronoun}. The pronoun \textit{who} and the personal pronouns change with the change of subject or object.\\
\xmpl
\indnt Who goes there?\\
\indnt He who is enlightened is sure to succeed.\\
A pronoun in a comparison is nominative if it is the subject of a stated or understood verb. We should avoid ``understood'' verbs by supplying them.
\xmpl
\indnt He plays better than I. \indnt \indnt He plays better than I do.


\section{Rule 11}\label{rl11}
When using a participial phrase at the beginning of a sentence, it must \textbf{refer to the grammatical subject}. Violating this rule may often make sentences ludicrous.\\
\xmpl
\indnt Approaching slowly to the crowded place, I found that an exhibition is going on.




\chapter{Elementary Principles of Composition}
Chapter \ref{ERU} documents out and fixes the common elementary rules that should be followed while constructing sound sentences. This chapter focuses on \emph{how to write sound passage} and make the writing more enjoyable.


\section{Choosing a Suitable Design} \label{CSD}
While writing, whether it's an essay, a formal writing, an article or a business purpose writing, a basic structural design is needed. But the writing should follow the writer's thoughts, but not necessarily in a certain order. So we need to determine the structure of our writing and stick with that structure.\\
While writing \emph{scientific papers}, the scientist knows how his writing content is organized and where he would use those, thus they know where they're headed while writing.


\section{Paragraph as the Unit Of Composition}
Whenever there're several topics to cover on a particular subject, each of the topic should be dealt with in a paragraph. Single sentences shouldn't be in a paragraph. But there can be exceptions which may indicate the relation between different parts of a same topic. In general, paragraphing requires good eye and a logical mind.\\
Enormous blocks of writing pushed together may look odd to read to the readers and this may discourage them to read at all. By breaking, this paragraph into pieces may help the readers to better understand it.


\section{Using Active Voice}
To write more direct, bold and concise sentence using active voice is preferred over passive voice. Using passive voice can even make the sentence indefinite, especially when `by...' is omitted. Whether to use active voice or passive voice is often determined by the need to make a particular word the subject of a sentence.\\
\xmpl
\indnt I did it. \emph{(Active)}\\
\indnt It was done (by me). \emph{(Passive)}


\section{Putting Statements in Positive Form}
We should use definite assertions. Whenever there're multiple ways to put a sentence, we should try to put it in a positive form and avoid using words like \emph{not}. Also placing negative and positive in opposition can make stronger structure. To avoid unnecessary doubt, auxiliaries like \emph{would, should, could, may, might, \emph{and} can} should be used only when real uncertainty is involved.
\xmpl
\indnt His proposal was disapproved.\\
\indnt I don't want pity or revenge but justice.\\
\indnt He never knew his affection for his country.\\


\section{Using Definite, Specific, Concrete Language}
While writing, it is better to be specific and to the point, rather than being general and vague about a topic. The readers are usually more attracted to this kind of writings. The writer must never be vague in exposition and in argument. We can see example from Herber Spencer's \textit{Philosophy of Style}:\\
The vague sentence---\\ \textit{In proportion as the manners, customs, and amusements of a nation are cruel and barbarous, the regulations of its penal code will be severe.}---can be turned into something particular in the following way---\\
\textit{In proportion as men delight in battles, bullfights, and combats of gladiators, will they punish by hanging, burning, and the rack.}\\


\section{Omitting Needless Words}
A sentence must not contain unnecessary words. This can be explained using an analogy--- just like a good snippet of \textit{code} doesn't need unnecessary commands, lines or operations--- a good sentence mustn't have words or lines that are not necessary.\\
\xmpl
\indnt Using `This' instead of `This is what....'
\indnt Using `disapproved' instead of `Not approved'\\


\section{Avoiding a Succession of Loose Sentences}
The loose sentences are sentences, consisting of two clauses, the second one being added by a conjunction. Loose sentences may be used to rectify the content of the topic and also to make the sentences easier, but using too many loose sentences may result in ambiguous empty writing, specially when the writer is not skilled enough.\\
\xmpl
\indnt He might consider paying the higher fees at a private university, if the teacher/student ratio is small, the teachers are highly qualified, and the job placement rate is high. \textit{A Loose Sentence}\\
He might consider paying the higher fees as a private university if the teacher/student ratio is small. Also the teachers should be highly qualified too and the job placement rate need to be high. \textit{Not A Loose Sentence}\\


\section{Expressing Coordinate Ideas in Similar Form}
The similar content and function of a particular topic should be outwardly similar. To do this the writer should group similar words and left out the dissimilar ones.\\
\xmpl
\indnt I saw a man on a hill with a telescope. (Not Coordinated)
\indnt I saw a man on a hill with my telescope. (Coordinated)\\


\section{Keeping Summaries in One Tense}
While summarizing something, a fixed tense should be kept throughout the whole summary. Shifting from one tense to another brings the appearance of uncertainty.


\section{Placing Emphatic Words at the End of a Sentence}
The proper place of the most important words the writer wants to convey is the end of the sentence. This rule equally applied to the paragraphs of a composition and articles.
\xmpl
\indnt Although the drug is highly effective, it has significant side effects.




\chapter{A Few Matters of Form}
The previous two chapters focus on how to construct sentences and paragraphs in writings. On the structures of the whole writing. Where to use exclamations, how to use hyphens, numerals, parenthesis, quotations etc, and how much margin should we have in our writings etc are discussed magnificently. 


\section{Colloquialisms}
If a colloquialism or a slang word or phrase is used, no quotation marks should be used.


\section{Exclamations}
Simple sentences should not be emphasized by using exclamation mark.\\
\xmpl
\indnt This is OK! \indnt \indnt This is OK.


\section{Headings}
If a manuscript is prepared for publication, a lot of space should be left at the top page 1 for editorial purposes. Placing the heading or title at least a fourth of the way down the page, a blank line should be left or its equivalent space after the heading. There should be no period after the heading, an exclamation or question mark may be used if the heading suits that.


\section{Hyphens}
A hyphen is usually needed when two or more words are combined to form a compound adjective and it shouldn't be used in words that can be written in a single word.\\
\xmpl
\indnt Hyper-text\indnt \indnt Hypertext.


\section{Margins}
Keeping righthand and lefthand margins equally wide is a good practice. However, only the lefthand margin should be used if the writings need lots of editing.


\section{Numerals}
Dates or other serial numbers should be written in figures or in Roman notations if they don't occur in dialogue.\\
\xmpl
\indnt September 19,1993 \indnt \indnt Chapter V Part III\\
\indnt ``Read verse twelve from chapter one'', He said.


\section{Parenthesis}
The sentence inside the parenthesis is usually punctuated as if no parenthesis is present. It is usually used to describe something in more detail.\\
\xmpl
\indnt When I reached home (around 7 PM), she was waiting for me.


\section{Quotations}
Quotations cited as evidence are usually introduced by a colon and enclosed quotation marks.\\
\xmpl
\indnt Here's what Arthur Conan Doyle said about it:``There is nothing more deceptive than an obvious fact.''\\
When a quotation is followed by an attribute phrase, the comma is enclosed within the quotation marks.\\
\xmpl
\indnt ``I wasn't there,'' he said.\\
Quotations started with \textit{that} are indirect and not used in quotation marks.\\
\xmpl
\indnt Leonard Cohen said that, poetry is just the evidence of life.\\


\section{Reference}
In general, references should be given in parenthesis or in footnotes, not in the body.


\section{Syllabication}
If a word is needed to be divided into pieces, we should use a dictionary to learn the syllables between which division should be made.


\section{Titles}
There're few rules about how the title should look like. The literary works's title should be italic with capitalized initials.\\
\xmpl
\indnt \textit{The Game of Thrones}



\chapter{Commonly Misused Words and Expressions}
In this chapter the writer picked out the commonly misused words and illustrated their actual uses. We misuse words unknowingly almost everyday, but this is a huge obstacle on the way to produce good writing. So the writer collected those words and tired to fit them in this book.
\begin{description}
	\item[All right] Used in the sense of ``Agreed''\\
	\xmpl
	\indnt There was none of that sure, all right look of the Americans about her.
	\item[Allude] It's often being confused with \textit{elude}. Allude means to \textit{refer}.\\
	\xmpl
	\indnt They know not the thing you distantly allude to.
	\item[Allusion] It's confused with \textit{Illusion} which means imaginary things, whereas allusion means to refer indirectly.\\
	\xmpl
	\indnt The penchant for classical allusion in Irish lyrics is not exclusive to this type of music. 
	\item[Among. Between] When more than two things are compared \textit{among} is used. \emph{Between} is used to compare two things.\\
	\xmpl
	\indnt And there's a mighty difference between a living thump and a dead thump.\\
	\indnt The bees were buzzing among the flowers.
	\item[Anybody] It refers to ``any person'' and must be written in one word.\\
	\xmpl
	\indnt Has anybody ever told you that you're beautiful?
	\item[As to whether] Shouldn't be used, instead just whether is enough.\\
	\xmpl
	\indnt Whether or not you take my deal, I'll help her.
	\item[As yet] Using \textit{yet} is almost always sufficient.\\
	\xmpl
	\indnt And yet, he came home after work every night.
	\item[Can] It means someone ``am(is,are) able'' and must not be used instead of just \textit{may}\\
	\xmpl
	\indnt I can take care of myself.
	\item[Care less] It is misused with the phrase ``couldn't care less''. If the \textit{n't} is disappeared, the meaning gets destroyed.\\
	\xmpl
	\indnt He had enough of those dirty lies, he couldn't care less anymore.
	\item[Certainly] If this word is used indiscriminately to intensify the statements, it can produce bad speech and if used in writing it'd be even worse.\\
	\xmpl
	\indnt Now we are certainly on the fuzzy edges, a place where words, often fuzzy in their meanings, begin to fail us.
	\item[Clever] It's a negative word and is often misused instead of ``intelligent''.\\
	\xmpl
	\indnt "You were pretty clever for a couple of kids," Dean said.
	\item[Literal. Literally] These words are often used incorrectly.\\
	\xmpl
	\indnt Guilt literally drowned him.\textit{(Wrong)}\\
	\indnt Guilt drowned him.\textit{(Correct)}\\
	\indnt ``Literally'' is literally one of the most misused words.
	\item[Personally] This word is often unnecessary.\\
	\xmpl
	\indnt Personally, I like being alone. \indnt \indnt I like being alone.
\end{description}


\chapter{An Approach to Style}
This chapter contains the advice of the writer to the reader about how they should write and what they should keep in mind while producing good writings.


\section{Placing Oneself in the Background}
A good practice in writing is placing oneself in the background, that is the writing should be independent of the writer's mood and temper. If the writing is good and solid these will eventually reflect in the writing. As someone becomes proficient in using language, his own writing style will emerge.


\section{Writing in a Natural Way}
It is good to write in a natural way, to use the words that comes naturally to someone. It might have some flaws but it's still good. On the contrary, it is bad to imitate other's writing style consciously.


\section{Working from A Suitable Design}
Before starting to write anything it is always a good idea to plan out the whole structure of the writing as mentioned in \ref{CSD}. Design may help the writer to compose what he wants in a more efficient way.


\section{Writing with Nouns And Verbs}
It is more preferred to write with verbs and nouns than adverbs and adjectives. The adjectives aren't able to help a weak or inaccurate noun.


\section{Revise And Rewrite}
Revising is an indispensable part of writings. Revising what's already written might give the writer ideas about how he can improve and also find out mistakes. So Revising and rewriting is a good way to improve writing.


\section{Avoid Overwriting}
Overwriting is one of the worst practices. It is always a good idea to avoid overwriting. When the writing is completed, it should be read and the excess should be deleted.


\section{Avoid Overstating}
Overstatement gives birth to suspicion into the mind of the readers. So it is always a good idea not to overstate, only state which is required to convey the message properly.


\section{Avoiding Usage of Qualifiers}
\textit{Rather, very, little, pretty}---these qualifiers should be avoided whenever possible as they make the writing debilitating. We should all try to be careful about using these.


\section{Not Affecting Breezy Manner}
The breezy style is often compared with works of an egocentric. So, it should be avoided.


\section{Using Orthodox Spelling}
When writing ordinary composition, it is always good to use the most orthodox spelling of a word.


\section{Avoid Unnecessary Explanations}
Unnecessary explanations may make the writing ambiguous. Besides, it may even annoy the readers. So it should be avoided. It is one of the common examples of bad writing.


\section{Constructing Awkward Adverbs}
As adverbs are easy to construct, this is misused very easily. It is easy to construct an adverb that sounds awkward. Using these type of adverb may reduce the value of the writing. So these should be avoided.\\
\xmpl
\indnt overly \indnt \indnt over\\
\indnt quitely \indnt \indnt quite\\


\section{Making Sure The Reader Knows The Speaker}
Using dialogue in writing goes to total failure if the speaker is not visible enough or the identity of the speaker becomes ambiguous.
So while writing dialogues, it is always advisable to keep in mind to identify the speaker clearly.


\section{Avoiding Fancy Words}
We should always try to stick to the easy words. Fancy words can easily create unnecessary ambiguity as well as make the writing inappropriate.


\section{Restricting The Use of Dialect}
Dialects often create confusion among the readers. As in dialect it is necessary to spell the words phonetically to capture the unusual inflections, if it is used then, it should be consistent.


\section{Being Clear}
Noting is worse than an ambiguous writing. Whatever the content of the writing is, being clear, to the point is always a good style.


\section{Not Injecting Opinion}
Without a good reason, injecting writer's opinion in the writing is a bad practice. As we all have our own opinion about something and may see something from a very different perspective, it is advisable not to use writer's own opinion in the writing.


\section{Using Figure of Speech Sparingly}
As readers may need time to understand something clearly, if figure of speeches like metaphors are misused, a huge misunderstanding may be created. It is specially true if metaphors are mixed up. So we need to be careful about this.


\section{Not Taking Shortcuts at The Cost of Clarity}
Using shortcuts to indicate something that may cost the clarity of the whole thing is always a bad practice. It is often seen while using shorthands to indicate names (example:Kentucky Fried Chicken --- KFC). It is better to use this kind of shorthand later in the writing but use the full version at the beginning.


\section{Avoiding Foreign Language}
There may come some time when the writer may need to borrow words from another language to express the writing more clearly. But it should be avoided whenever possible. 


\section{Preferring The Standard to The Offbeat}
While writing we should always use the standard method. Using vocabularies not popular or which doesn't have widespread uses should be avoided to make the writing unambiguous for most people. Using the words only widespread among certain generation is also discouraged as the theme of using words changes in almost every generations.



\chapter{Conclusion}
\textit{The Elements of Style} is certainly one of the best book available today to learn about \textit{how to write}. After studying the book thoroughly, it should be possible to build up an elegant writing style for any individual, given that he keeps in mind about the do's and don'ts mentioned and discussed in this book.

\end{document}