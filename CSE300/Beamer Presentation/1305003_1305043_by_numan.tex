\documentclass[9pt]{beamer}
\usetheme{keynote}
\usepackage{enumitem}
\usepackage{multicol}
\usepackage{multirow}
\usepackage{graphics}


\title{Comparison of Color Feature for Image Retrival}
\author{S.R. Kodituwakku \inst{1} \and S.Selvarajah \inst{2}}


\institute{\inst{1}Department of Statistics \& Computer Science, University of Peradeniya \and %
               		  \inst{2} Department of Physical Science, Vavuniya Campus, University of Jaffna}

\date{\parbox{\linewidth}{\centering%
  \today\endgraf\medskip
  Presented By\endgraf\bigskip
  Abdullah Al Mamun \hspace*{3cm} S. Mahmudul Hasan\endgraf\medskip
  1305003 \hspace*{4.5cm} 		  1305043\endgraf\bigskip
  Dept.\ of Computer Science \& Engineering (CSE), \endgraf\medskip
  Bangladesh University of Engineering and Technology (BUET)}}






\begin{document}

\begin{frame}
\titlepage
\end{frame}


\begin{frame}
\frametitle{Outline}
\tableofcontents
\end{frame}






\section{Methodology}

\begin{frame}
\fontsize{18pt}{30}\selectfont
\centering
METHODOLOGY
\end{frame}

\subsection{Color Reduction}
\begin{frame}
\frametitle{Color Reduction}
\pause
True Color: supports 24-bit for three RGB colors\\[\baselineskip]
\pause
Total number of possible colors: $2^{24}$\\[\baselineskip]
\pause
Reduced to: 256
\end{frame}

\subsection{Histogram Based Image Retrival}
\begin{frame}
\frametitle{Histogram Based Image Retrival}
Two techniques:\\[\baselineskip]
\pause
\begin{description}
\item[GCH] Global Color Histogram
\pause
\item[LCH] Local Color Histogram\\[\baselineskip]
\pause
\end{description}
GCH:\\ Represents images with \textbf{single} histogram.\\[\baselineskip]
\pause
LCH:\\ Images are divided into fixed blocks of \textbf{size 8x8} \& histogram is obtained for \textbf{each} block
\end{frame}



\begin{frame}
\frametitle{Histogram Based Image Retrival}
Steps for using histogram:\\
\pause
\begin{itemize}[label=$\star$]
\setlength\itemsep{2em}

\item Histograms of all images in the databes are computed \& stored
\pause
\item Histogram of the query image is computed
\pause
\item Measure similarity of database's images and query images using euclidian distance metrics
\pause
\item Identify relevant images using a fixed threshold value\\[\baselineskip]
\end{itemize}
\pause
\textbf{Euclidian Distance Metrics?}\\[\baselineskip]
\pause
For \textbf{3D}
\begin{equation}
distance((x,y,z),(a,b,c)) = \sqrt{(x-a)^2+(y-b)^2+(z-c)^2}
\end{equation}
\pause
Let,
Image A: $(x_1, x_2,\ldots,x_n)$\\
Image B: $(y_1, y_2,\ldots,y_n)$\\[\baselineskip]
\pause
Distance between A \& B = $\sqrt{\sum_{i=1}^{n}(x_i-y_i)^2}$
\end{frame}

\subsection{Extraction of Visual Feature}
\begin{frame}
\frametitle{Extraction of Visual Feature}
\pause
Viusal features are extracted using color histograms and moments in equations...\textbf{MAMUN}\\[\baselineskip]
\pause
\begin{itemize}[label=$\star$]
\setlength\itemsep{2em}

\item Mean and std. deviations of database images are calculated and stored
\pause
\item Database's images's mean and std. deviations are compared with that of query image
\pause
\item Relevant images are ranked by using a fixed threshold value of the difference between the properties being compared\\[\baselineskip]
\pause
\end{itemize}
\end{frame}

\subsection{Coherence Bassed Image Retrival}
\begin{frame}
\frametitle{Coherence Measure}
Pixels: \textbf{Coherent} or \textbf{Incoherent}\\[\baselineskip]
\pause
Coherent pixels:\\
\pause
\begin{itemize}[label=$\star$]
\item Part of a sizable contiguous region
\pause
\item Pixel groups are determined by computing \textbf{connected components} using \textbf{8} connected neighbours within a color bucket
\pause
\item Size of connected components need to exceed a fixed value (1\% of the total image area)\\[\baselineskip]
\pause
\end{itemize}
Incoherent pixels: The rest of the pixels\\[\baselineskip]
\pause
The color space of the image is discretized to \textbf{3 distinct color (R,G,B)}.\\[\baselineskip]
\pause
Pixels within a color bucket is classified as coherent or incoherent.\\[\baselineskip]
\pause
A \textbf{color coherence vector} is used to represent this classification for each color.\\[\baselineskip]
\pause
Relevant images are retrived by computing the similarity between the \textbf{query vector} and \textbf{image vectors}.

\end{frame}

\begin{frame}
\fontsize{18pt}{30}\selectfont
\centering
RESULTS
\end{frame}


\section{Results}
\begin{frame}
\frametitle{Results}
\pause
General purpose image database containing 14500 images is used.\\[\baselineskip]
\pause
\textbf{Categories: }\\[\baselineskip]
\pause
\begin{multicols}{4}
    \begin{itemize}[label=$\blacksquare$]
        \item Africans \& villages
        \item Beaches
        \item Buildings
        \item Buses
        \item Dinosaurs
        \item Elephants
        \item Flowers
        \item Horses
        \item Mountains \& glaciers
        \item Foods
        \item Faces
        \item Objects
        \item Drawings
        \item Textures
        \item Natural scenes\\[\baselineskip]
    \end{itemize}
\end{multicols}
\pause
Image Format:\textbf{ JPEG}\\
Imgae Size: \textbf{$384x256$}\\
Image Representation: \textbf{RGB color space}\\[\baselineskip]
\pause
\textbf{Five} different image of each catergory are used as query image some having \textbf{uniform} color distribution, some having \textbf{non-uniform} color distribution and other having \textbf{average} color distribution\\

\end{frame}




\begin{frame}
\frametitle{Results}

\pause
\begin{minipage}{.9\textwidth}
\begin{table}
\caption{\textbf{Category - Dinosaurs}}

\resizebox{.5\columnwidth}{!}{
\begin{tabular}{|c|c|c|c|c|}
\hline
Descriptor & Recall & Precision\\
\hline
\multirow{2}{*}{\textbf{CM}} & 0.100137931 & 0.698748797\\
& 0.09862069 & 0.754219409\\
& 0.097586207 & 0.815092166\\
\hline
\multirow{2}{*}{\textbf{GCH}} & 0.099586207 & 0.529325513\\ & 0.09862069 & 0.580121704\\ & 0.096551724 & 0.615114236\\
\hline
\multirow{2}{*}{\textbf{LCH}} & 0.097586207 & 0.5\\ & 0.095586207 & 0.508064516\\
& 0.093586207 & 0.534672971\\
\hline
\multirow{2}{*}{\textbf{CCV}} & 0.10062069 & 0.555597867\\ & 0.099586207 & 0.585801217\\ & 0.096551724 & 0.631483987\\
\hline
\end{tabular}
}
\end{table}
\end{minipage}

\pause
\begin{minipage}{.9\textwidth}
\begin{table}
\caption{\textbf{Category - Africans}}

\resizebox{.5\columnwidth}{!}{
\begin{tabular}{|c|c|c|c|c|}
\hline
Descriptor & Recall & Precision\\
\hline
\multirow{2}{*}{\textbf{GCH \& CCV}} & 0.088531187 & 0.871287129\\
& 0.08249497 & 0.872340426\\
& 0.079476861 & 0.877777778\\
\hline
\multirow{2}{*}{\textbf{LCH \& CCV}} & 0.100603622 & 0.487804878\\
& 0.099597586 & 0.5\\
& 0.097585513 & 0.510526316\\
\hline
\multirow{2}{*}{\textbf{CCV, CM, LCH \& GCH}} & 0.095573441 & 0.95959596\\
& 0.09054326 & 0.97826087\\
& 0.088531187 & 0.988764045\\
\hline
\multirow{2}{*}{\textbf{GCH, LCH \& CCV}} & 0.098591549 & 0.439461883\\
& 0.096579477 & 0.507936508\\
& 0.095573441 & 0.50802139\\
\hline
\end{tabular}
}
\end{table}
\end{minipage}
\end{frame}



\begin{frame}
\fontsize{18pt}{30}\selectfont
\centering
DECISIONS
\end{frame}


\section{Decisions}
\begin{frame}
\frametitle{Decisions}
\pause
No feature is superior to other as performance is color distribution dependent.\\[\baselineskip]
\pause
\begin{description}
\item[Uniform Color Distribution] Color histogram gives better performance
\pause
\item[Average Color Distribution] Color moments gives better performance
\pause
\item[Widely Scattered Colors] CCV gives better performance\\[\baselineskip]
\end{description}
\pause
The \textbf{combination} of different descriptors gives most satisfactory results.
\end{frame}

\begin{frame}
\fontsize{18pt}{30}\selectfont
\centering
Thank you!\\[\baselineskip]
\pause
Questions?
\end{frame}


\end{document}